\documentclass[10pt, a4paper]{scrartcl}
\usepackage[utf8]{inputenc}
\usepackage[english,russian]{babel}
\usepackage{indentfirst}
\usepackage{misccorr}
\usepackage{graphicx}
\usepackage{amsmath}
\usepackage{hyperref}
\usepackage{geometry} \geometry{verbose,a3paper,tmargin=2cm,bmargin=2cm,lmargin=2.5cm,rmargin=1.5cm}
\numberwithin{equation}{section}
\large
\begin{document}
	\tableofcontents 
	\newpage
	\section{Циклические группы}
	\noindent Рассматривая множество $Z_{n} = {0, 1, \ldots, n - 1}$ с операциями сложения и умножения по модулю $n$, элементы этого множества будем заключать в квадратные скобки и отмечать нижним индексом $n$, т.е. если $k, m \in Z$, то $[k]_{n},[m]_{n} \in Z_{n}$, и арифметические операции над $[k]_{n}$ и $[m]_{n}$ выполняются по модулю $n$.
	\\
	
	 \textbf{Теорема А.1} \textit{Любая конечная циклическая группа порядка n изоморфна группе ($Z_{n}, +$).}
	 \\
	 
	 Доказательство. Пусть конечная циклическая группа $G$ порядка $n$ порождается элементом $g$. Рассмотрим изображение $f \; : \; g^k \rightarrow k$. Тогда для любых элементов $g_{1} = g^k$ и $g_{2} = g^m$ группы $G$ имеем:
	 \begin{equation}
	  f(g1g2) = f(g^k g^m) = f(g^{k+m}) = [k+m]_{m} = [k]_{n} + [m]_{n} = f(g^k) + f(g^m) = f(g1) + f(g2)
	 \end{equation}
	 
	 Таким образом $f$ -- изоморфизм. Теорема доказана. 
	 \\
	 
	 \textbf{Теорема А.2} \textit{Пусть $m$ делит $n$. В любой конечной циклической группе порядка $n$ существует единственная подгруппа порядка $m$. Все подгруппы циклической группы циклические.}
	 \\
	 
	 Доказательство. В силу теоремы А.1, достаточно показать, что доказываемое свойство справедливо для ($Z_{n}, +$). Прежде всего отметим, что по крайней мере одна подгруппа порядка $m$ существует. Эта подгруппа образована числами, делящимися на $d = n/m:0,d, \ldots,(m-1)d$, порождается элементом $d$, и поэтому является циклической. Отсутствие других подгрупп следует из того, что для каждого элемента $a$ такой подгруппы должно выполняться равенство $m \cdot a = 0 \; (mod \; $n$)$, т.е. каждый ее элемент должен делиться на $d$. В $Z_{n}$ таких чисел ровно $m$, и все они принадлежат первой подгруппе. Теорема доказана. 
	 \\
	 
	 \textbf{Теорема А.3} \textit{В любой циклической группе порядка $n$ существует ровно $\varphi(n)$ порождающих элементов.} 
	 \\
	 
	 Доказательство. Снова воспользуемся теоремой А.1. Покажем, что в ($Z_{n}, +$) порождающими элементами будут все числа взаимнопростые с $n$ и только они. Известно, что взаимная простота чисел $n$ и $a$ эквивалентна существованию таких целых $k$ и $m$, что $kn + ma = 1$. Из этого равенства следует, что 
	 \begin{equation}
	  m \cdot a = 1 \; (mod \; n).
	 \end{equation}
	 Поэтому единица, порождающий элемент в ($Z_{n}, +$), является $m$-й степенью элемента $a$. Следовательно, $a$ -- порождающий элемент в ($Z_{n}, +$). С другой стороны, если $a$ -- порождающий элемент, то единица должна быть степенью $a$, т.е. должно существовать целое $m$, для которого справедливо (А.1), откуда в свою очередь следует взаимная простота $n$ и $a$. Теорема доказана. 
	 
	 Простым следствием двух предыдущих теорем является следующее утверждение. 
	 \\
	 
	 \textbf{Теорема А.4} \textit{Пусть $m$ делит $n$. В любой конечной циклической группе порядка $n$ существует $\varphi(m)$ элементов порядка $m$.}
	 \\
	 
	 Доказательство. Каждый элемент порядка $m$ порождает подгруппу такого же порядка. Так как в циклической группе есть только одна подгруппа порядка $m$, то все элементы порядка $m$ принадлежат одной и той же подгруппе и являются её порождаюшими элементами. Теорема доказана. 
	 
	 Автор работы : https://vk.com/idmegafeil ( Гусев Елисей) 
\end{document}